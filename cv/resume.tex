\documentclass[margin,line]{resume}
\newif\ifReferences
\newif\ifOnline
\usepackage{hyperref}
\usepackage{graphicx}
\usepackage[skins]{tcolorbox}
\usepackage{everypage}
\usepackage{fancyhdr} % Added for page numbering

%Make bottom margin nicer
\addtolength{\textheight}{-1.2in}

% Page numbering setup
\pagestyle{fancy}
\fancyhf{}
\renewcommand{\headrulewidth}{0pt}
\fancyfoot[C]{\thepage} % Page number in the center of the footer


\begin{document}


\Referencesfalse
\Onlinefalse


\name{\Large Antonio Ken Iannillo, Ph.D.}
\begin{resume}

%\tikz\node[circle,draw,inner sep=1.5cm,fill overzoom image=AKI_PHOTO] (A) {};
%\hfill\begin{tikzpicture}[radius=2cm,delta angle=180]
%\path[draw,thick,fill overzoom image=AKI_PHOTO]
%  (0,0) arc [start angle=-90] -- ++(-3,0) arc [start angle=90] -- cycle;
%\end{tikzpicture}

\section{\mysidestyle Contact Information}
\ifOnline
	mail: ak.iannillo@gmail.com	 							\hfill Luxembourg\\
\else
    5 Rue St Mathieu				                   			\hfill cell: +352 661 196 166\\
    2138 Luxembourg								        \hfill mail: ak.iannillo@gmail.com\\
    Luxembourg										\hfill website: \url{akiannillo.github.io/}\\
\fi




\section{\mysidestyle Research Interests}
Cybersecurity, Software Engineering, Secure and Compliant Data Management;
\textit{application domains:} Health Tech, Blockchain, Logistics, Trusted Execution Environments, AI

\section{\mysidestyle Education}
\textbf{Universit\`a degli Studi di Napoli Federico II}, Naples, Italy \vspace{1mm}\\%
\textsl{PhD in Computer Engineering} \hfill \textbf{November 2014 -- October 2017}\vspace{-3mm}\\\vspace{-1mm}%
\begin{list2}
%	\item Defence: 31 January 2018
	\item Thesis: \textsl{Dependability assessment of Android OS}
	\item Advisors: Dr. Domenico Cotroneo and Dr. Roberto Natella
\end{list2}

\textbf{Universit\`a degli Studi di Napoli Federico II}, Naples, Italy \vspace{1mm}\\%
\textsl{M.Sc., Computer Engineering} \hfill \textbf{November 2011 -- January 2014}\vspace{-3mm}\\\vspace{-1mm}%
\begin{list2}
	\item GPA: 3.91
	\item Thesis: \textsl{A Fault injection tool for Java software applications}
	\item Final grade: 110/110 cum laude
\end{list2}\vspace{-1.5mm}    


\textbf{Universit\`a degli Studi di Napoli Federico II}, Naples, Italy \vspace{1mm}\\%
\textsl{B.Sc., Computer Engineering} \hfill \textbf{October 2008 -- October 2011}\vspace{-3mm}\\\vspace{-1mm}%
\begin{list2}
	\item GPA: 3.88
	\item Thesis: \textsl{Comparison between programming models in Facebook and Google Plus}
        \item Final grade: 110/110 cum laude
\end{list2}\vspace{-1.5mm}    


\section{\mysidestyle Professional Experience}

\textbf{WAVY MEET S.à r.l.}, Luxembourg City, Luxembourg \hfill \textbf{March 2023 -- present}\\
\textit{CEO and Co-Founder}\hfill\\
\vspace{-3mm}\\\vspace{-1mm}
\begin{list2}
	\item \filbreak WAVY MEET platform: Born from ``OWL - FNR JUMP project'', it's a medical software platform with a secure and comprehensive approach to remote rehabilitation.
	\item \filbreak\textbf{FIT4START \#13 graduate:} 
	
	\emph{Selected in October 2022. Graduated in June 2023. Acquired budget: 150,000 euros.}
	
	One of the 20 selected startups, from a total of 214 applications coming from 42 countries, for specialized coaching organized by LuxInnovation. WAVY MEET graduated from the program FIT4START funded by the Ministry of the Economy as planned.
	
	\item \filbreak\textit{website:} \url{https://wavymeet.com/}.
\end{list2}


\textbf{University of Luxembourg}, Luxembourg City, Luxembourg\\  \phantom{.} \hfill \textbf{November 2018 -- November 2023}\\
\textit{Interdisciplinary Centre for Security, Reliability and Trust}\hfill\\
\textit{Research Associate - since November 2018}\hfill\\
\textit{Research Scientist - since February 2020}\hfill  
\vspace{-3mm}\\\vspace{-1mm}
\begin{list2}
	\item \filbreak {Research group:} SEDAN headed by Dr. Radu State.
	\item \filbreak {Ph.D. students support}: Dedicated to the mentorship and support of SEDAN Ph.D. students, my role involves providing guidance and assistance to the next generation of researchers. This includes sharing expertise, facilitating research discussions, and fostering a collaborative and productive research environment within the group. 
	\item \filbreak\textbf{Luxembourgish R\&D fund (partner: GULLIVER)}: 
	
	\emph{On-going project. Acquired budget: 800,000 euros. Total project budget: 3,000,000 euros.}
	
	I acquired a partnership with Gulliver Luxembourg S.à r.l., securing a budget of 800,000 million euros for SnT in a granted project from the Ministry of Economy using the Luxembourgish R\&D fund. As the Project Manager and Scientist, I am actively involved among SnT (project partner) and Gulliver Luxembourg S.à r.l. (project leader). The primary objective is to enhance software used in transport and logistic services through the incorporation of novel AI technologies, thereby improving efficiency and reliability in the industry.

	\item \filbreak\textbf{Industrial partnership (partner: QUANTSTAMP)}: 
	
	\emph{On-going project. Acquired budget: 260,000 euros.}
	
	I successfully acquired a partnership with Quantstamp, securing a budget of 260,000 euros. Leading the SnT team in collaboration with Quantstamp, spearheading initiatives to advance the security of decentralized finance (DeFi) on the blockchain. This industrial project plays a crucial role in addressing emerging challenges in the blockchain space, contributing to the secure and reliable implementation of DeFi systems. 
		
	\item \filbreak\textbf{CONCORDIA - EU H2020:} 
	
	\emph{Project ended in December 2023. Managed budget: 465,000 euros. Total project budget: 23,000,000 euros.}
	
	I was the leader for liaisons with stakeholders. CONCORDIA was a cybersecurity competence network with leading research, technology, industrial, and public competences. I was actively involved in organizing annual events and several cybersecurity workshops to foster collaboration and knowledge exchange among experts in the field. More information about the project can be found at \url{https://www.concordia-h2020}.
	
	\item \filbreak\textbf{OWL - FNR JUMP project}: 
	
	\emph{Project ended in June 2023. Acquired budget: 250,000 euros.}
		
	Served as the Principal Investigator for the "Online Workout Platform" (OWL) project, acquiring a budget of 250,000 euros funded by the FNR. The role involved steering the project towards its goal of establishing a startup company dedicated to commercializing a tele-rehabilitation platform designed for cardiac patients. This initiative not only contributes to scientific research but also has the potential to make a positive impact on healthcare by providing innovative solutions for rehabilitation. 
	
	\item \filbreak\textbf{STARTS - FNR Junior Core:} 
	
	\emph{Project ended in December 2020. Acquired budget: 550,000 euros.}
	
	As the Principal Investigator of the "SecuriTy Assessment of tRusTzone-m based Software" (STARTS) project, successfully acquired a budget of 550,000 euros. The focus was on developing a comprehensive methodology for the security assessment of software based on TrustZone-M technology. Additionally, working on a novel verification and validation framework to implement this methodology effectively. This project represents a significant contribution to advancing the field of software security. More information about the project can be found at \url{https://starts.uni.lu/}. 

	\item \filbreak\textbf{Cybersecurity in Robot (partner: Ministry of Defence)}: 
	
	\emph{Project ended in June 2020.}	
		
	I led a project focusing on the cybersecurity assessment and enforcement in a Robotic Operating System (ROS) for military and defense applications. Utilized Trusted Execution Environments to enhance security measures. Project ended in June 2020.



\end{list2}



\textbf{Universit\`a degli Studi di Napoli Federico II}, Naples, Italy\\ \phantom{.} \hfill \textbf{November 2017 -- October 2018}\\
\textbf{Research Fellow, Dependable Systems and Software Engineering Research Team}\hfill 
\vspace{-3mm}\\\vspace{-1mm}
\begin{list2}
	\item \filbreak\textit{Automatic Feature Extraction and Analysis of Faulty Code:} Software faults are code imperfections that may lead to the system's eventually failure. A deep understanding of the code developers insert specific software faults into will help several tasks such as bug prevention, bug detection, and software fault injection.
    \item \filbreak\textit{Fuzz Testing on Android OS:} Study and research on important challenges for the robustness (security and dependability) of the Android OS. Study and research on evolutionary algorithms and search strategies. Design and development of a smart testing tool on Android.
\end{list2}

\textbf{Critiware s.r.l.}, Naples, Italy\hfill\textbf{November 2017 - April 2018}\\
\textbf{Research Consultant}\hfill 
\vspace{-3mm}\\\vspace{-1mm}
\begin{list2}
	\item \filbreak\textit{Python Fault Injection:} Study and research on Python parsing technologies and programming language theory. Design and implementation of a DSL framework for code changes in Python code. Collaboration with Huawei Technologies Co. Lts.
\end{list2}

\textbf{Northeastern University}, Boston, MA\hfill\textbf{September 2016 -- April 2017}\\
\textbf{Research Assistant (Visitor), Network and Distributed Systems Security Lab}\hfill 
\vspace{-3mm}\\\vspace{-1mm}
\begin{list2}
	\item \filbreak\textit{Vendor customizations on Android system services:} Study and research on important challenges for the robustness (security and dependability) of the Android OS. Design and development of an innovative testing tool. Robustness and security testing on physical devices. Tutored by Dr. Cristina Nita-Rotaru.
\end{list2}

\textbf{Consorzio Interuniversitario Nazionale Italiano (CINI)}, Naples, Italy\\
\null\hfill\textbf{January 2014 - October 2014}\\
\textbf{Junior Research Fellow}\hfill 
\vspace{-3mm}\\\vspace{-1mm}
\begin{list2}
	\item \filbreak\textit{NFVI reliability:} Research of new approach for software reliability evaluation of virtualized environments for Network Function Virtualization (NFV). Design and implementation of a reliability evaluation tool for VMWare ESXi. Collaboration with Huawei Technologies Co. Lts.
	\item[] \item[] \hfill \textit{Professional experience section continues on next page...}
	\item \filbreak\textit{PON SVEVIA:} Study and research of usability for fault injection tools. Design and implementation of an integrated fault injection tool (Eclipse plug-in) for the fault injection test design and analysis of results, in Java and C/C++ software.
\end{list2}

\filbreak
\textbf{R\&D department, Infosys LTD}, Bangalore, India\hfill
\textbf{June 2014 -- September 2014}\\
\textbf{R\&D Intern}\hfill 
\vspace{-3mm}\\\vspace{-1mm}
\begin{list2}
    \item \filbreak\textit{Java Fault Injection:} Study and research of new approaches for the injection of software defects. Design and development of a tool for fault injection into the Java Bytecode. This thesis work has been conducted in India during the preparation of my MSc. degree thesis. Tutored by Dr. Santonu Sarkar.
\end{list2}

\section{\mysidestyle Honors and Awards}
ISSRE 2017 Conference Best Paper Award
\vspace{1mm}\\%
Netsoft 2015 Conference Best Paper Award
\vspace{1mm}\\%
Information Technology and Electrical Engineering PhD 2014-2017 scholarship by Universit\`a degli Studi di Napoli Federico II

\filbreak
\section{\mysidestyle Relevant\\Skills} 
\textbf{Programming:} Python, Javascript, Bash, Java, C/C++\vspace{2mm}\\
\textbf{Development Tools:} Git, GitHub, GitLab, JIRA, Confluence, Slack, Eclipse, VS Code, Android Studio, Maven, Gradle, Jenkins, Docker \vspace{2mm}\\
\textbf{Cybersecurity:} TLS, SSL, SSH, OWASP ZAP, Burp Suite, Kali Linux, Metasploit, F\reflectbox{R}IDA  \vspace{2mm}\\
\textbf{Research:} ns-3, SimPy, R, Python (NumPy, Pandas), Scikit-learn, TensorFlow, Matplotlib  \vspace{2mm}\\
\textbf{Project Management}: Agile (Scrum), Gantt charts, WBS, Risk Management, Software Lifecycle Management, ISO knowledge \vspace{2mm}\\
\textbf{Interpersonal Skills}: Strong communication, teamwork, leadership, adaptability \vspace{2mm}\\

\filbreak
\section{\mysidestyle LANGUAGES} 
Italian (Native), English (Professional), French (Basic), Spanish (Basic)

\filbreak
\section{\mysidestyle Publications}

\filbreak
Bahareh Parhizkari, Antonio Ken IANNILLO, Christof FERREIRA TORRES, Sebastian Banescu, Joseph Xu\\
``\textbf{Timely Identification of Victim Addresses in DeFi Attacks}''
[7th International Workshop on Cryptocurrencies and Blockchain Technology (CBT 2023)]
\filbreak
Antonio Ken Iannillo, Sean Rivera, Darius Suciu, Radu Sion, and Radu State\\
``\textbf{An REE-independent Approach to Identify Callers of TEEs in TrustZone-enabled Cortex-M Devices}''
[ACM CyberPhysical System Security Workshop (CPSS ’22)]
\filbreak
Christof Ferreira Torres, Antonio Ken Iannillo, Arthur Gervais, and Radu State\\
``\textbf{ConFuzzius: A Data Dependency-Aware Hybrid Fuzzer for Smart Contracts}''
[Security and Privacy (EuroS\&P), 2021 IEEE European Symposium on]
\filbreak
Christof Ferreira Torres, Antonio Ken Iannillo, Arthur Gervais, and Radu State\\
``\textbf{The Eye of Horus: Spotting and Analyzing Attacks on Ethereum Smart Contracts}''
[Financial Cryptography and Data Security (FC), 2021 25th International Conference on]
\filbreak
Sean Rivera, Antonio Ken Iannillo, Sofiana Lagraa, Clement Joly and Radu State\\
``\textbf{ROS-FM: Fast Monitoring for the Robotic Operating System(ROS)}''
[Engineering of Complex Computer Systems (ICECCS), 2020 25th International Conference on]
\filbreak
Sean Rivera, Vijay K. Gurbani, Sofiane Lagraa, Antonio Ken Iannillo, and Radu State\\
``\textbf{Leveraging eBPF to preserve user privacy for DNS, DoT, and DoH queries}''
[Availability, Reliability, and Security (ARES), 2020 15th International Conference on]
\filbreak
Antonio Ken Iannillo and Radu State\\
``\textbf{A Proposal for Security Assessment of Trustzone-M based Software}''
[Software Reliability Engineering (ISSRE), 2019 IEEE 30th International Symposium on]
\filbreak
Domenico Cotroneo, Luigi De Simone, Antonio Ken Iannillo, Roberto Natella, Stefano Rosiello***\\
``\textbf{Analyzing the Context of Bug-Fixing Changes in the OpenStack Cloud Computing Platform}''
[Software Reliability Engineering (ISSRE), 2019 IEEE 30th International Symposium on]
\filbreak
Sean Rivera, Sofiane Lagraa, Antonio Ken Iannillo, Radu State\\
``\textbf{Auto-encoding Robot State against Sensor Spoofing Attacks}''
[Software Reliability Engineering (ISSRE), 2019 IEEE 30th International Symposium on]
\filbreak
Domenico Cotroneo, Antonio Ken Iannillo, Roberto Natella***\\
``\textbf{Evolutionary Fuzzing of Android OS Vendor System Services}''
[Empirical Software Engineering Journal (2019)]
\filbreak
Antonio Ken Iannillo, Roberto Natella, Domenico Cotroneo, Cristina Nita-Rotaru\\
``\textbf{Chizpurfle: A Gray-Box Android Fuzzer for Vendor Service Customizations}''
[Software Reliability Engineering (ISSRE), 2017 IEEE 28th International Symposium on]\\\textbf{Conference Best Paper Award}
\filbreak
Domenico Cotroneo, Francesco Fucci, Antonio Ken Iannillo, Roberto Natella, Roberto Pietrantuono***\\
``\textbf{Software Aging Analysis of the Android Mobile OS}''
[Software Reliability Engineering (ISSRE), 2016 IEEE 27th International Symposium on]
\filbreak
Domenico Cotroneo, Antonio Ken Iannillo, Roberto Natella, Roberto Pietrantuono, Stefano Russo***\\
``\textbf{The Software Aging and Rejuvenation Repository}''
[Software Reliability Engineering (ISSRE), 2015 IEEE 26th International Symposium on]
\filbreak
Domenico Cotroneo, Luigi De Simone, Antonio Ken Iannillo, Anna Lanzaro, Roberto Natella***\\
``\textbf{Dependability Evaluation and Benchmarking of Network Function Virtualization Infrastructures}''
[Network Softwarization (NetSoft), 2015 IEEE 1st Conference on]\\\textbf{Conference Best Paper Award}
\filbreak
Domenico Cotroneo, Luigi De Simone, Antonio Ken Iannillo, Anna Lanzaro, Roberto Natella***\\
``\textbf{Usability of Fault Injection}''
[Software Reliability Engineering (ISSRE), 2014 IEEE 25th International Symposium on]
\filbreak
Domenico Cotroneo, Luigi De Simone, Antonio Ken Iannillo, Anna Lanzaro, Roberto Natella, Jian Fan, Wang Ping\\
``\textbf{Network Function Virtualization: Challenges and Directions for Reliability Assurance}''
[Software Reliability Engineering (ISSRE), 2014 IEEE 25th International Symposium on]
\filbreak
{\footnotesize*** international authorship order does not apply; authors are in an alphabetic order, due to internal policies applied by the authors' research group.}






%\filbreak
%\section{\mysidestyle Relevant Courses}
%Security Analytics, Machine Learning, Data Mining,
%Distributed Systems, Networking, Operating Systems,
%Data Structures, Algorithms, Differential Privacy, 
%Cryptography, Quantum Information

%\filbreak
%\section{\mysidestyle Activities}
%Organize group meetings for Networks and Distributed Systems Security PhD students\vspace{1.5mm}\\
%Endurance running\vspace{1.5mm}\\
%Organize and compete in tournaments for Super Smash Brothers\vspace{1.5mm}\\
 
%\filbreak
%\section{\mysidestyle Professional Affiliations}


%\filbreak
%\section{\mysidestyle References}
%\ifReferences
%	\begin{tabular}{@{}p{6cm}p{6cm}}
%	 \textbf{Dr. Cristina Nita-Rotaru} 	& \textbf{Dr. Sonia Fahmy}\\
%	 Associate Professor					& Professor\\
%	 Northeastern University				& Northeastern University\\
%	 Boston, MA			                	& Boston, MA \\
%	 phone: 765-496-6757					& phone: 765-494-6183\\
%	 email: crisn@cs.purdue.edu				& email: fahmy@cs.purdue.edu\\
%	 \end{tabular}
%
%\else % References
%	{\sl Available on request}
%\fi
\end{resume}
\end{document}
